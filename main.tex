\documentclass{minimal}
\usepackage[utf8]{inputenc}
\usepackage[T1]{fontenc}

\usepackage[default]{lato}
\usepackage{raleway}
\usepackage{fontawesome}
\usepackage[dvipsnames]{xcolor}
\usepackage{soul}

\usepackage{hyperref}
% \usepackage{showframe}

\usepackage[a4paper, tmargin=0.2in, bmargin=0.3in, lmargin=0.3in, rmargin=0.3in]{geometry}
\usepackage{lipsum}
\usepackage{contour}
\usepackage{ulem}
\usepackage{tcolorbox}
\usepackage{varwidth}
\usepackage{pbox}
\usepackage[super]{nth}
\usepackage{ocgx}
\usepackage{lipsum}

% Fancy underline: basically, draw underline under phantom text, then draw real text over it (overlapping the phantom) with white contour, so the contour goes over the underline insted of intersecting it, making "gaps" in underline for letters which have parts below baseline
\renewcommand{\ULdepth}{1.8pt}
\contourlength{0.8pt}
\newcommand{\prettyUline}[1]{% phantom and countour (both!) operate in restricted horizontal environment, i.e. don't support line breaks; todo fix it sometime 
  \uline{\phantom{#1}}%
  \llap{\contour{white}{#1}}%
}


\definecolor{dark-gray}{gray}{0.25}
\definecolor{very-dark-gray}{gray}{0.15}
%
% CONSTANTS
%
\def\headerFont{\fontfamily{Raleway-TLF}\fontsize{36}{40}}
\def\titleFont{\fontfamily{Raleway-TLF}\fontsize{24}{28}}
\def\subtitleFont{\fontfamily{Raleway-TLF}\fontsize{18}{22}}
\def\miniSubtitleFont{\fontfamily{Raleway-TLF}\fontsize{16}{19}\color{dark-gray}}
\def\dateFont{\fontfamily{Raleway-TLF}\fontsize{15}{17}\scshape}
\def\miniDateFont{\fontfamily{Raleway-TLF}\fontsize{13}{15}\color{dark-gray}\scshape}
\def\ribbonFont{\fontfamily{Raleway-TLF}\fontsize{12}{14}}
\def\contentFont{\fontsize{11}{13}}
\def\largeContentFont{\fontsize{13}{15}}
\def\rSubtitleFont{\fontfamily{Raleway-TLF}\fontsize{14}{16}}
\def\rSubSubtitleFont{\fontfamily{Raleway-TLF}\fontsize{13}{15}}

% link formating I use through CV
\def\linkText#1{\color{very-dark-gray}\prettyUline{#1}}
\def\cvLink#1#2{\href{#1}{\linkText{#2}}} % doesn't support line breaks — keep link text short!


% various font weights
\def\extraLight{\fontseries{el}\selectfont}
\def\light{\fontseries{l}\selectfont}
\def\semiBold{\fontseries{sb}\selectfont}
\def\bold{\fontseries{b}\selectfont}
\def\medium{\fontseries{m}\selectfont}
\def\alignRight{\null\hfill}

% margins and other spaces
\def\iconsTopMargin{-15mm}
\def\iconsSpaceBetweenRows{-3mm}
\def\iconsMinipageWidth{0.333\linewidth}
\def\iconSpaceBetweenText{1mm}
\def\alignRight{\hspace*{\fill}}
\def\leftColWidth{0.6\linewidth}
\def\rightColWidth{0.35\linewidth}
\def\titleBottomMargin{0mm}
\def\dateTopMargin{-3mm}
\def\miniSubtitleTopMargin{3mm}
\def\twoColMiniSubtitleTopMargin{-10mm}
\def\contentTopMarginAlt{1mm}
\def\subsectionSpace{4mm}
\def\contentTopMargin{-7mm}
\def\sectionTopmargin{7mm}
\def\columnTopMargin{-3mm}
\def\rContentTopMargin{-7mm}
\def\pageTwoTopMargin{0mm}
\def\ribbonTopMargin{-3mm}
\def\ribbonSpace{1mm}
\def\ribbonBottomMargin{-1mm}


\newtcolorbox{ribbonBox}[1]{colframe=#1!75!black, colback=white, hbox, boxsep=0mm, left=0.4mm, right=0.4mm, text height=0.8em}
\def\ribbon#1#2{\pbox{5cm}{\begin{ribbonBox}{#1}{\ribbonFont\medium #2}\end{ribbonBox}}\hspace{\ribbonSpace}}

\def\ribbonJava{\ribbon{orange}{Java}}
\def\ribbonAndroid{\ribbon{green}{Android}}
\def\ribbonPlay{\ribbon{lime}{Play! Framework}}
\def\ribbonPostgres{\ribbon{violet}{PostgreSQL}}
\def\ribbonSwing{\ribbon{brown}{Swing}}
\def\ribbonJavascript{\ribbon{yellow}{Javascript}}
\def\ribbonR{\ribbon{blue}{R}}
\def\ribbonTidyverse{\ribbon{purple}{Tidyverse}}
\def\ribbonSpringBoot{\ribbon{OliveGreen}{Spring Boot}}
\def\ribbonKotlin{\ribbon{red}{Kotlin}}
\def\ribbonFfmpeg{\ribbon{olive}{ffmpeg}}
\def\ribbonLatex{\ribbon{teal}{\LaTeX}}
\def\ribbonFontAwesome{\ribbon{cyan}{Font Awesome}}
\def\ribbonThymeleaf{\ribbon{ForestGreen}{Thymeleaf}}

\begin{document}\noindent
%
% HEADER (title)
%
\begin{center}
{\headerFont\extraLight LUKA \light JOVIČIĆ}
\end{center}
%
% Social links
%
\begin{center}
\begin{minipage}[b]{\iconsMinipageWidth} %I're using minipages for horizontal alignment in the whole document
\faHome\hspace{\iconSpaceBetweenText}Pančevo, Serbia
\end{minipage}%
\begin{minipage}[b]{\iconsMinipageWidth}
\begin{center}
\faEnvelope\hspace{\iconSpaceBetweenText}\href{mailto:luka.jovicic16+cv@gmail.com}{luka.jovicic16@gmail.com}
\end{center}
\end{minipage}%
\begin{minipage}[b]{\iconsMinipageWidth}
\alignRight\faGlobe\hspace{\iconSpaceBetweenText}\href{https://poly.work/lukaj}{poly.work/lukaj} %I stopped updating this at some point, I should probably continue sometime (but for now I find LaTeX more fun)
\end{minipage}\newline%
%
\vspace{\iconsSpaceBetweenRows}
\begin{minipage}[b]{\iconsMinipageWidth}
\faGithub\hspace{\iconSpaceBetweenText}\href{https://github.com/luka-j}{github.com/luka-j}
\end{minipage}%
\begin{minipage}[b]{\iconsMinipageWidth}
\begin{center}
\faStackOverflow\hspace{\iconSpaceBetweenText}\href{https://stackoverflow.com/u/2363015}{stackoverflow.com/u/2363015}
\end{center}
\end{minipage}%
\begin{minipage}[b]{\iconsMinipageWidth}
\alignRight\faLinkedin\hspace{\iconSpaceBetweenText}\href{https://linkedin.com/in/luka-j}{linkedin.com/in/luka-j} %it makes sense that linkedin comes front and center, but I find this arrangement aesthetically more pleasing: stackoverflow is the longest link, so it goes below the longest link in the first row
\end{minipage}\newline
\end{center}

\vskip -0.7cm  % this whole header is negative vspace clusterfuckery. Just don't touch anything.
\noindent
\hspace{-0.33in}
\rule{1.1\linewidth}{0.2mm}
%
% LEFT COLUMN
%
%
% Experience
%
\begin{minipage}[t]{\leftColWidth}\vspace{\columnTopMargin}
\begin{minipage}[t]{\linewidth}\vspace{\sectionTopmargin}
{\titleFont\light EXPERIENCE}\linebreak

\vspace{\titleBottomMargin}
{\subtitleFont\bold JUNIOR SOFTWARE ENGINEER \extraLight\scshape | Centili}\newline
\vfill\vspace{\dateTopMargin}{\dateFont\extraLight Oct 2019 -- present\alignRight}\linebreak\newline

\vspace{\twoColMiniSubtitleTopMargin}
\begin{minipage}[t]{0.99\linewidth}
{\miniSubtitleFont\bold SOFTWARE ENGINEER INTERN\\\miniDateFont\extraLight Aug -- Oct 2020 }
\end{minipage}
\newline
%\begin{minipage}[t]{0.4\linewidth}
%{\miniSubtitleFont\bold INTERN\\\miniDateFont\extraLight Aug -- Oct 2019 }
%\end{minipage}\newline

\vspace{\contentTopMarginAlt} % this is quite an unique box because this two-col split for Junior and Intern. While interning I was mostly working on API Gateway, later the time was split between API Gateway and maintaining and upgrading other products (analytics, alerting, circuit breaker, containerization, http client and internal utils lib). After that I got reassigned to the strike team working on new projects.
{\contentFont As a part of the core engine team, my work usually consists of maintaining and upgrading the heart of the platform which processes over 1.5 million transactions daily. My main projects are building an API Gateway system from ground up and containerizing and upgrading the CI/CD pipeline. I've enhanced analytics collection and worked on migrating from RMI to HTTP communication between services. I was also a part of a strike team which developed new microservices and adapted existing infrastructure for emerging business needs. I've had the chance to work with Cloudflare and its serverless Workers platform to cut loading time of payment pages for up to 2 seconds.}
\end{minipage}\newline% I don't want any kind of space here
%
\begin{minipage}[t]{\linewidth}\vspace{\subsectionSpace}
{\subtitleFont\bold WIKI AMBASSADOR \extraLight\scshape | Wikimedia Serbia}\newline
{\vfill\vspace{\dateTopMargin}\dateFont\extraLight Dec 2018 -- present\alignRight}\linebreak\newline

\vspace{\contentTopMargin}
{\contentFont As a student Wiki ambassador, I’ve worked with other Wikimedia volunteers, established new partnerships between Wikimedia Serbia and teachers and professors, envisioned student projects which contribute to Wikipedia and Wikidata, hosted workshops, and overseen and assisted in students’ work.}
\end{minipage}\newline%
%
% Formal Education
%
\begin{minipage}[t]{\linewidth}\vspace{\sectionTopmargin}
{\titleFont\light FORMAL EDUCATION}\newline

\vspace{\titleBottomMargin}
{\subtitleFont\bold FACULTY OF MATHEMATICS \vspace{0.75mm}\newline\extraLight\scshape University of Belgrade}\newline

{\vspace{-3mm}\dateFont\extraLight BSc Informatics, 240ECTS | Oct 2018 -- present}\newline

\vspace{-3mm}
{\contentFont I'm currently finishing \nth{3} year of Bachelor studies in Informatics, passing all exams on schedule. %You can find all my coursework on GitHub. % you can't, TODO upload and add link
During the course of my studies, I've developed a configurable ballistic \cvLink{https://github.com/luka-j/ballistic-projectile/}{flight path simulator} in Python. I was also a part of the team which developed \cvLink{https://github.com/pearpanda/computer-graphics-project/}{bouncing ball animation} in pure OpenGL}.
\end{minipage}\newline%
%
\begin{minipage}[t]{\linewidth}\vspace{\subsectionSpace}
{\subtitleFont\bold MATHEMATICAL GRAMMAR SCHOOL}\newline
{\vfill\vspace{\dateTopMargin}\dateFont\extraLight Sep 2014 -- Jun 2018\alignRight}\linebreak\newline

\vspace{\contentTopMargin}
{\contentFont I graduated from a school with High National Distinction status, specialized for mathematics and related disciplines, with 5/5 grade. My \cvLink{https://www.mg.edu.rs/uploads/files/images/stories/dokumenta/maturski/luka-jovicic.pdf}{graduation thesis} on data scraping and analysis of national primary school graduation results was selected as \cvLink{https://www.mg.edu.rs/sr/nastava/najbolji-maturski-radovi}{one of the best informatics theses} for school year 2017/18.}
\end{minipage}\newline%
%
% Extracurricular activities
%
\begin{minipage}[t]{\linewidth}\vspace{\sectionTopmargin}
{\titleFont\light EXTRACURRICULARS}\newline

\vspace{\titleBottomMargin}
{\subtitleFont\bold CS WEEK AT MGB \dateFont\extraLight(2018)}\newline
\newline

\vspace{-7mm}
{\contentFont I held a lecture on exploratory data analysis at 2018 edition of \cvLink{http://csnedelja.mg.edu.rs/}{Computer Science Week at Mathematical Grammar School} for selected students. I've built a demo web app written in R \cvLink{https://github.com/luka-j/csw5-eda}{available on my GitHub}.}
\end{minipage}
\newline%
%
\begin{minipage}[t]{\linewidth}\vspace{\subsectionSpace}
{\subtitleFont\bold ENGLISH LANGUAGE CONFERENCES}\newline
\newline

\vspace{-7mm}
{\contentFont I've co-authored and presented papers (with high school teacher and peers) on Canadian film and teaching English as a second language at two international language conferences in \cvLink{https://www.npao.ni.ac.rs/filozofski-fakultet/download/391_964a4088cf25e02b20de75ae7a784ddf}{Belgrade (2017)} and \cvLink{https://press.um.si/index.php/ump/catalog/book/360}{Celje (2018)} respectively.}
\end{minipage}
% End left column
\end{minipage}%
%
%
% RIGHT COLUMN
%
%
\hfill
\begin{minipage}[t]{\rightColWidth}\vspace{\columnTopMargin}
%
% Tech
%
\begin{minipage}[t]{\linewidth}\vspace{\sectionTopmargin}
{\alignRight\titleFont\light TECH STACK}\linebreak\newline

\vspace{\contentTopMargin}
{\contentFont\bold Main toolbox: \medium Java, Spring Boot, Docker, Jenkins, PostgreSQL, Kotlin} %\medium screws over my fonts, but in this case it actually looks okay: it's a bit bigger than the rest

\vspace{2mm}
{\contentFont\bold Familiar with: \medium C, Bash, Python, R, Play! Framework, Android, WildFly, JavaScript, HTTP, \LaTeX}

\vspace{2mm}
{\contentFont\bold Experimenting with: \medium Scala, Redis \& Sentinel, Docker Swarm, distributed computing, Prolog, C++}

\vspace{2mm}
{\contentFont\bold Curious about: \medium Rust, Go, big data, GCP, neural networks}
\end{minipage}\newline%
%
% Informal education
%
\begin{minipage}[t]{\linewidth}\vspace{\sectionTopmargin}
{\alignRight\titleFont\light INFORMAL\linebreak\alignRight EDUCATION}\linebreak\newline
{\alignRight\rSubtitleFont\bold Petnica Summer Insitute}\newline
{\alignRight\rSubSubtitleFont\light Machine learning, 2019}\linebreak\newline

\vspace{\rContentTopMargin}
{\contentFont I've participated in an intensive 10-day bootcamp on machine learning, organized by Microsoft Development Center Serbia and Petnica SC. After partaking in lectures and workshops, I've worked on a project which used a conditional generative adversarial network to generate metal album cover arts. Our testimonial is featured \cvLink{http://psiml.petnica.rs/MetalGan.php}{on the PSI:ML website}.}

\end{minipage}\newline%
%
\begin{minipage}[t]{\linewidth}\vspace{\subsectionSpace}
{\alignRight\rSubtitleFont\bold Petnica Science Center}\newline
{\alignRight\rSubSubtitleFont\light Psychology seminars, 2017–18}\linebreak\newline

\vspace{\rContentTopMargin}
{\contentFont I've taken part in various lectures and workshops about research psychology. After writing practice papers and projects, I spent one year working on my own research on the role of emoticons and emojis in nonverbal irony, presented at the annual “Step Into Science” conference and published in Petnica Papers.}
\end{minipage}\newline%
%
\begin{minipage}[t]{\linewidth}\vspace{\subsectionSpace}
{\alignRight\rSubtitleFont\bold JCon}\newline
{\alignRight\rSubSubtitleFont\light Online conference, 2020}\linebreak\newline

\vspace{\rContentTopMargin}
{\contentFont I've attended one of the largest Java conferences, learning about the future of Java, trends in microservice development in 2020, best testing practices and pathways to cloud native apps. \cvLink{https://gist.github.com/luka-j/2c1230dea9f24c3a9b9cb30d661bb02e}{My notes are available on Gist.}} % I should probably clean up those notes
\end{minipage}\newline%
%
% Scholarships
%   
\begin{minipage}[t]{\linewidth}\vspace{\sectionTopmargin}
{\alignRight\titleFont\light SCHOLARSHIPS}\linebreak\newline

\vspace{\rContentTopMargin}
{\contentFont\textbf{National scholarship for students}\linebreak (2019/2020, 2020/2021)  by Ministry of Education, awarded to higher education students which maintain exceptional GPA (9.0 or higher out of 10) and have maximum study efficiency.}\alignRight\linebreak
\end{minipage}\newline
% End right column
\end{minipage}
%
% End page 1
%
%
% PAGE 2
%
% Left column
\begin{minipage}[t]{\leftColWidth}\vspace{\pageTwoTopMargin}
%
% Selected projects
%
\begin{minipage}[t]{\linewidth}
{\titleFont\light SELECTED PROJECTS}\newline

\vspace{\titleBottomMargin}
{\subtitleFont\bold STORIES \extraLight\scshape| Library \& Android app \href{https://github.com/luka-j/stories}{\faGithub} \href{https://play.google.com/store/apps/details?id=rs.lukaj.android.stories}{\faAndroid}}\newline%There's no \faGooglePlay :/

\vspace{\ribbonTopMargin}
\ribbonJava \ribbonAndroid \ribbonPlay \ribbonPostgres\newline
\vspace{\ribbonBottomMargin}

{\contentFont App for creating, sharing and playing choose-your-adventure stories. I’ve designed a simple procedural language for storytelling and implemented a platform-agnostic interpreter (open source). Android app (proprietary) features a code editor, a visual story editor and an online marketplace. It won Special prize for innovativity at Telekom Srbija Regional App Challenge 2018.}
\end{minipage}\newline%
%
\begin{minipage}[t]{\linewidth}\vspace{\subsectionSpace}
{\subtitleFont\bold UPIS SUITE \extraLight\scshape| Desktop \& Web apps \href{https://github.com/luka-j/UpisScraper}{\faGithub} \href{https://github.com/luka-j/UpisStats}{\faGithub} \href{https://github.com/luka-j/UpisDesktop}{\faGithub}}\newline

\vspace{\ribbonTopMargin}
\ribbonJava \ribbonPlay \ribbonSwing \ribbonPostgres \ribbonJavascript\newline
\vspace{\ribbonBottomMargin}

{\contentFont Three apps designed to work together in order to collect and interpret data about Serbian primary schools. I've built a scraper which collects students’ grades and graduation data, enrolment simulator with support for custom rules, SQL-like data plotting language, web frontend for plots, performant plotter in Swing (able to handle and modify over $10^5$ points) and a visual query builder for desktop. Data and code is freely available on my GitHub.}
\end{minipage}\newline%
%
\begin{minipage}[t]{\linewidth}\vspace{\subsectionSpace}
{\subtitleFont\bold CHAT SURVEYS \extraLight\scshape| Web apps \href{https://github.com/Runtime-T-error}{\faGithub}}\newline

\vspace{\ribbonTopMargin}
\ribbonR \ribbonTidyverse \ribbonPostgres \ribbonJava \ribbonSpringBoot\newline
\vspace{\ribbonBottomMargin}

{\contentFont A set of apps which transform surveys to conversational format, using Facebook messages as proof of concept. My role was focused on interpreting existing engagement data and building dashbord for future analytics in R's shiny framework, supporting simple plots as well as inference methods such as linear regression and PCA. The project won \nth{3} prize at RAF Hakaton 2019.}
\end{minipage}\newline%
%
\begin{minipage}[t]{\linewidth}\vspace{\subsectionSpace}
{\subtitleFont\bold VIDEO COMPRESSOR \extraLight\scshape| Web app \href{https://github.com/luka-j/MediaCompressor}{\faGithub} \href{https://compressor.luka-j.rocks/}{\faGlobe}}\newline

\vspace{\ribbonTopMargin}
\ribbonKotlin \ribbonSpringBoot \ribbonPostgres \ribbonFfmpeg\newline
\vspace{\ribbonBottomMargin}

{\contentFont A web app which transcodes videos using x265 codec and applies size optimizations for video and audio data. It can use one or more worker nodes to speed up the process, as well as act as a worker for configured master nodes.}
\end{minipage}\newline%
%
\begin{minipage}[t]{\linewidth}\vspace{\subsectionSpace}
{\subtitleFont\bold CV \extraLight\scshape| Interactive document \href{https://github.com/luka-j/CV/tree/latex-v3}{\faGithub}}\newline

\vspace{\ribbonTopMargin}
\ribbonLatex \ribbonFontAwesome \newline
\vspace{\ribbonBottomMargin}

{\contentFont I've designed, made-up and built this very document on top of \LaTeX's \texttt{minimal} document class. It makes use of interactive PDF features which are not universally supported (e. g. in browsers), but falls back gracefully.}
\end{minipage}\newline
%
% EVEN MORE PROJECTS
%
\begin{minipage}{\linewidth}\vspace{5mm}
{\titleFont\light MORE PROJECTS}\vspace{1mm}\newline
{\subtitleFont \light CLICK THE \switchocg{HelpText}{\faChevronCircleRight}}\newline

\vspace{\titleBottomMargin}
{\subtitleFont\bold NOTEKEEPER \extraLight\scshape| Android app \href{https://play.google.com/store/apps/details?id=rs.luka.android.studygroup}{\faAndroid} \switchocg{NotekeeperText}{\faChevronCircleRight}}\newline

\vspace{\ribbonTopMargin}
\ribbonJava \ribbonAndroid \ribbonPlay \ribbonPostgres\newline
\end{minipage}\newline%
%
\begin{minipage}{\linewidth}
{\subtitleFont\bold\scshape MinimalHttpClient \extraLight\scshape| Java library \href{https://github.com/luka-j/MinimalHttpClient}{\faGithub} \switchocg{HttpClientText}{\faChevronCircleRight}}\newline

\vspace{\ribbonTopMargin}
\ribbonJava\newline
\end{minipage}\newline%
%
\begin{minipage}{\linewidth}
{\subtitleFont\bold LAB INVENTORY \extraLight\scshape| Web app \href{https://github.com/luka-j/laboratorija-inventar}{\faGithub} \switchocg{LabInventoryText}{\faChevronCircleRight}}\newline

\vspace{\ribbonTopMargin}
\ribbonKotlin \ribbonSpringBoot \ribbonPostgres \ribbonThymeleaf \newline
\end{minipage}\newline%
%
\begin{minipage}{\linewidth}
{\subtitleFont\bold\scshape BG BUS \extraLight\scshape| Android app \href{https://github.com/luka-j/BgBus}{\faGithub} \switchocg{BgBusText}{\faChevronCircleRight}}\newline

\vspace{\ribbonTopMargin}
\ribbonJava \ribbonAndroid \newline
\end{minipage}\newline%
% End left column
\end{minipage}
%
%
% RIGHT COLUMN
%
%
\hfill
\begin{minipage}[t]{\rightColWidth}\vspace{\pageTwoTopMargin}
%
% Competitons
%
\begin{minipage}[t]{\linewidth}
{\alignRight\titleFont\light COMPETITIONS}\linebreak\newline

{\alignRight\rSubtitleFont\bold RAF Hackathon \light (team, 2020)}\newline
{\alignRight\largeContentFont\medium \nth{3} prize for \cvLink{https://github.com/Runtime-T-error}{Chat Surveys}}\newline

% {\alignRight\rSubtitleFont\bold WATFOI API Design \alignRight Workshop \light (2020)}\newline
% {\alignRight\largeContentFont\medium \nth{1} prize for API design and spec}\newline
%
{\alignRight\rSubtitleFont\bold Telekom Srbija Regional App \alignRight Challenge \light (2018)}\newline
{\alignRight\largeContentFont\medium Special prize for \cvLink{https://play.google.com/store/apps/details?id=rs.lukaj.android.stories}{Stories}}\newline

{\alignRight\rSubtitleFont\bold RAF Challenge \light (2017)}\newline
{\alignRight\largeContentFont\medium Finalist with \cvLink{https://github.com/luka-j/UpisDesktop}{UpisDesktop}}\newline

% {\alignRight\rSubtitleFont\bold MatHackathon \light (team, 2017)}\newline
% {\alignRight\largeContentFont\medium Finalist with an Arduino smart \alignRight home app}\newline
%
{\alignRight\rSubtitleFont\bold mt:s App Competition \light (2016)}\newline
{\alignRight\largeContentFont\medium \nth{2} prize for \cvLink{https://play.google.com/store/apps/details?id=rs.luka.android.studygroup}{Notekeeper}}\newline

{\alignRight\rSubtitleFont\bold mt:s App Competition \light (2015)}\newline
{\alignRight\largeContentFont\medium \nth{2} prize for \cvLink{https://github.com/luka-j/BgBus}{BgBus}}\newline

{\alignRight\rSubtitleFont\bold National programming \alignRight competitions \light (2014--2017)}\newline
{\alignRight\largeContentFont\medium Qualified to the national level}\newline

{\alignRight\rSubtitleFont\bold National programming \alignRight competitions \light (2012--2014)}\newline
{\alignRight\largeContentFont\medium Qualified to Serbian Informatics \alignRight Olympiad}\newline

{\alignRight\rSubtitleFont\bold National programming \alignRight competitions \light (2011/12)}\newline
{\alignRight\largeContentFont\medium \nth{3} prize at National competition}
\end{minipage}\newline%
%
\begin{minipage}[t]{\linewidth}\vspace{\sectionTopmargin}
{\alignRight\titleFont\light INTERESTS}\linebreak\newline

\vspace{\contentTopMargin}
{\contentFont\bold Design: \medium I especially enjoy document design, be it a CV or a \cvLink{https://luka-j.github.io/petnica/poster.pdf}{research poster}. I can usually find my way around Adobe InDesign and Illustrator.} %\medium screws over my fonts, but in this case it actually looks okay: it's a bit bigger than the rest

\vspace{2mm}
{\contentFont\bold Creative writing: \medium When not writing computer programs or technical documentation, I tend to write a story or two. In 2016, my short story ``Trapped in the Dollhouse'' won \nth{2} prize at \cvLink{http://www.englishbook.rs/aktuelnosti/konkursi/97-pobednici-konkursa-my-english-book-2016}{My English Book competition}.}

\vspace{2mm}
{\contentFont\bold Philosophy, science, law: \medium Tired of tech talk? Strike up a conversation about paradox of tolerance, statistical inference, or state sovereignty and I'll follow!}

\vspace{-2mm}
\rule{\linewidth}{0.1mm}
{\parbox{\linewidth}{\color{white}\fontsize{6.5}{8}\selectfont Hi there! You must be really bored or really interested reading small white text. Let me help: drop me a message and we can talk!}}
\end{minipage}
%
% More project descriptions
%
\noindent
\rlap{\begin{ocg}{OCG 0}{HelpText}{0} %so basically I use OCG so these things pop up only when appropriate arrow is clicked; I also want all of these boxes to overlap, so I'm using rlap to make them use no space and overlap (but to have their content visible, on click of course). 
  {\parbox{\linewidth}{\contentFont Great, it works! You can use the \faChevronCircleRight \ next to a project name to see its description. Click the same arrow again to close. Or you can use this one: \hideocg{HelpText}{\faChevronCircleLeft}}}
\end{ocg}\newline}
\rlap{\begin{ocg}{OCG 1}{NotekeeperText}{0}
\noindent
  {\parbox{\linewidth}{\contentFont Productivity and collaboration app for groups of students attending the same school. Provides a streamlined way for collecting and sharing notes, questions and multimedia content for different courses. Includes course organizer and exam calendar. The app won \nth{2} prize at Sixth mt:s App competition (2016).\hideocg{NotekeeperText}{\faChevronCircleLeft}}}
\end{ocg}\newline}
\rlap{\begin{ocg}{OCG 2}{HttpClientText}{0}
\noindent
  {\parbox{\linewidth}{\contentFont Implementation of HTTP/1.1 protocol from the ground up, using plain TCP sockets in Java. Provides an option to manually configure connection pool and reuses sockets wherever appropriate. Exposes both blocking (with timeout) and non-blocking (callback-based) API. Provides common abstractions for sending and receiving textual, raw and file content. Transparently handles redirects, content and transfer encodings.\hideocg{HttpClientText}{\faChevronCircleLeft}}}
\end{ocg}\newline}
\rlap{\begin{ocg}{OCG 3}{LabInventoryText}{0}
\noindent
  {\parbox{\linewidth}{\contentFont Web app for managing laboratory inventory, used in a local hospital. Tailored specifically for their needs, it tremendously speeds up inventory tracking and reporting for medical and non-medical supplies. UI was specifically built to be intuitive to non-technical users and print-friendly. \hideocg{LabInventoryText}{\faChevronCircleLeft}}}
\end{ocg}\newline}
\rlap{\begin{ocg}{OCG 4}{BgBusText}{0}
\noindent
  {\parbox{\linewidth}{\contentFont Android app which simplifies navigation in Belgrade on foot and using public transport. It is built upon a few custom heuristic algorithms which take into account time of day, type of transport and how often it is deployed. After algorithms are run with different parameters, four best results are displayed to the user and drawn on the map. The app was awarded \nth{2} prize at Fifth mt:s App Competition (2015). \hideocg{BgBusText}{\faChevronCircleLeft}}}
\end{ocg}\newline}

\vspace{6mm}
\rule{\linewidth}{0.1mm}
{\contentFont Curious about even more projects? Check out \cvLink{https://github.com/luka-j?tab=repositories}{my GitHub} and \cvLink{https://poly.work/lukaj}{Polywork!}}
\end{minipage}
\end{document}
