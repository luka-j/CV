\documentclass{minimal}
\usepackage[utf8]{inputenc}
\usepackage[T1]{fontenc}

\usepackage[default]{lato}
\usepackage{raleway}
\usepackage{fontawesome}
\usepackage[dvipsnames]{xcolor}
\usepackage{soul}

\usepackage{hyperref}
% \usepackage{showframe}

\usepackage[a4paper, tmargin=0.2in, bmargin=0.3in, lmargin=0.3in, rmargin=0.3in]{geometry}
\usepackage{lipsum}
\usepackage{contour}
\usepackage{ulem}
\usepackage{tcolorbox}
\usepackage{varwidth}
\usepackage{pbox}
\usepackage[super]{nth}
\usepackage{ocgx}
\usepackage{lipsum}

\renewcommand{\ULdepth}{1.8pt}
\contourlength{0.8pt}

\newcommand{\prettyUline}[1]{% phantom and countour (both!) operate in restricted horizontal environment, i.e. don't support line breaks; todo fix it sometime 
  \uline{\phantom{#1}}%
  \llap{\contour{white}{#1}}%
}


\definecolor{dark-gray}{gray}{0.25}
\definecolor{very-dark-gray}{gray}{0.15}
%
% CONSTANTS
%
\def\headerFont{\fontfamily{Raleway-TLF}\fontsize{36}{40}}
\def\titleFont{\fontfamily{Raleway-TLF}\fontsize{24}{28}}
\def\subtitleFont{\fontfamily{Raleway-TLF}\fontsize{18}{22}}
\def\miniSubtitleFont{\fontfamily{Raleway-TLF}\fontsize{16}{19}\color{dark-gray}}
\def\dateFont{\fontfamily{Raleway-TLF}\fontsize{15}{17}\scshape}
\def\miniDateFont{\fontfamily{Raleway-TLF}\fontsize{13}{15}\color{dark-gray}\scshape}
\def\ribbonFont{\fontfamily{Raleway-TLF}\fontsize{12}{14}}
\def\contentFont{\fontsize{12}{14}}
\def\largeContentFont{\fontsize{13}{15}}
\def\rSubtitleFont{\fontfamily{Raleway-TLF}\fontsize{14}{16}}
\def\rSubSubtitleFont{\fontfamily{Raleway-TLF}\fontsize{13}{15}}

\def\linkText#1{\color{very-dark-gray}\prettyUline{#1}}
\def\cvLink#1#2{\href{#1}{\linkText{#2}}} % doesn't support line breaks — keep link text short!


\def\extraLight{\fontseries{el}\selectfont}
\def\light{\fontseries{l}\selectfont}
\def\semiBold{\fontseries{sb}\selectfont}
\def\bold{\fontseries{b}\selectfont}
\def\medium{\fontseries{m}\selectfont}
\def\alignRight{\null\hfill}

\newtcolorbox{ribbonBox}[1]{colframe=#1!75!black, colback=white, hbox, boxsep=0mm, left=0.4mm, right=0.4mm, text height=0.8em}
\def\ribbon#1#2{\pbox{5cm}{\begin{ribbonBox}{#1}{\ribbonFont\medium #2}\end{ribbonBox}}\hspace{\ribbonSpace}}

\def\iconsTopMargin{-15mm}
\def\iconsSpaceBetweenRows{-3mm}
\def\iconsMinipageWidth{0.333\linewidth}
\def\iconSpaceBetweenText{1mm}
\def\alignRight{\hspace*{\fill}}
\def\leftColWidth{0.6\linewidth}
\def\rightColWidth{0.35\linewidth}
\def\titleBottomMargin{0mm}
\def\dateTopMargin{-3mm}
\def\miniSubtitleTopMargin{3mm}
\def\twoColMiniSubtitleTopMargin{-10mm}
\def\contentTopMarginAlt{1mm}
\def\subsectionSpace{4mm}
\def\contentTopMargin{-7mm}
\def\sectionTopmargin{7mm}
\def\columnTopMargin{-3mm}
\def\rContentTopMargin{-7mm}
\def\pageTwoTopMargin{0mm}
\def\ribbonTopMargin{-3mm}
\def\ribbonSpace{1mm}
\def\ribbonBottomMargin{-1mm}

\def\ribbonJava{\ribbon{orange}{Java}}
\def\ribbonAndroid{\ribbon{green}{Android}}
\def\ribbonPlay{\ribbon{lime}{Play! Framework}}
\def\ribbonPostgres{\ribbon{violet}{PostgreSQL}}
\def\ribbonSwing{\ribbon{brown}{Swing}}
\def\ribbonJavascript{\ribbon{yellow}{Javascript}}
\def\ribbonR{\ribbon{blue}{R}}
\def\ribbonTidyverse{\ribbon{purple}{Tidyverse}}
\def\ribbonSpringBoot{\ribbon{OliveGreen}{Spring Boot}}
\def\ribbonKotlin{\ribbon{red}{Kotlin}}
\def\ribbonFfmpeg{\ribbon{olive}{ffmpeg}}
\def\ribbonLatex{\ribbon{teal}{\LaTeX}}
\def\ribbonFontAwesome{\ribbon{cyan}{Font Awesome}}
\def\ribbonThymeleaf{\ribbon{ForestGreen}{Thymeleaf}}

\begin{document}\noindent
%
% HEADER
%
\begin{center}
{\headerFont\extraLight LUKA \light JOVIČIĆ}
\end{center}
%
% Social links
%
\begin{center}
\begin{minipage}[b]{\iconsMinipageWidth}
\faHome\hspace{\iconSpaceBetweenText}Pančevo, Serbia
\end{minipage}%
\begin{minipage}[b]{\iconsMinipageWidth}
\begin{center}
\faEnvelope\hspace{\iconSpaceBetweenText}\href{mailto:luka.jovicic16+cv@gmail.com}{luka.jovicic16@gmail.com}
\end{center}
\end{minipage}%
\begin{minipage}[b]{\iconsMinipageWidth}
\alignRight\faGlobe\hspace{\iconSpaceBetweenText}\href{https://luka-j.github.io}{luka-j.github.io}
\end{minipage}\newline%
%
\vspace{\iconsSpaceBetweenRows}
\begin{minipage}[b]{\iconsMinipageWidth}
\faGithub\hspace{\iconSpaceBetweenText}\href{https://github.com/luka-j}{github.com/luka-j}
\end{minipage}%
\begin{minipage}[b]{\iconsMinipageWidth}
\begin{center}
\faStackOverflow\hspace{\iconSpaceBetweenText}\href{https://stackoverflow.com/u/2363015}{stackoverflow.com/u/2363015}
\end{center}
\end{minipage}%
\begin{minipage}[b]{\iconsMinipageWidth}
\alignRight\faLinkedin\hspace{\iconSpaceBetweenText}\href{https://linkedin.com/in/luka-j}{linkedin.com/in/luka-j}
\end{minipage}\newline
\end{center}

\vskip -0.7cm  % this whole header is negative vspace clusterfuckery. Just don't touch anything.
\noindent
\hspace{-0.33in}
\rule{1.1\linewidth}{0.2mm}
%
% Left column
%
%
% Experience
%
\begin{minipage}[t]{\leftColWidth}\vspace{\columnTopMargin}
\begin{minipage}[t]{\linewidth}\vspace{\sectionTopmargin}
{\titleFont\light EXPERIENCE}\linebreak

\vspace{\titleBottomMargin}
{\subtitleFont\bold SOFTWARE ENGINEER \extraLight\scshape | Centili}\newline
\vfill\vspace{\dateTopMargin}{\dateFont\extraLight Aug 2019 -- present\alignRight}\linebreak\newline

\vspace{\twoColMiniSubtitleTopMargin}
\begin{minipage}[t]{0.5\linewidth}
{\miniSubtitleFont\bold JUNIOR\\\miniDateFont\extraLight Oct 2019 -- Dec 2020 }
\end{minipage}
\begin{minipage}[t]{0.4\linewidth}
{\miniSubtitleFont\bold INTERN\\\miniDateFont\extraLight Aug -- Oct 2019 }
\end{minipage}\newline

\vspace{\contentTopMarginAlt}
{\contentFont As a part of the core engine team, my work usually consisted of maintaining and upgrading the heart of the platform which processes over 1.5 million transactions daily. My main projects were building an API Gateway system from ground up and containerizing, upgrading and maintaining our CI/CD pipeline. I was also a part of a strike team which developed new microservices and adapted existing infrastructure for emerging business needs.}
\end{minipage}\newline%
%
\begin{minipage}[t]{\linewidth}\vspace{\subsectionSpace}
{\subtitleFont\bold WIKI AMBASSADOR \extraLight\scshape | Wikimedia Serbia}\newline
{\vfill\vspace{\dateTopMargin}\dateFont\extraLight Dec 2018 -- present\alignRight}\linebreak\newline

\vspace{\contentTopMargin}
{\contentFont As a student Wiki ambassador, I’ve worked with other Wikimedia volunteers, established new partnerships between Wikimedia Serbia and teachers and professors, envisioned student projects which contribute to Wikipedia and Wikidata, hosted workshops, and overseen and assisted in students’ work.}
\end{minipage}\newline%
%
% Formal Education
%
\begin{minipage}[t]{\linewidth}\vspace{\sectionTopmargin}
{\titleFont\light FORMAL EDUCATION}\newline

\vspace{\titleBottomMargin}
{\subtitleFont\bold FACULTY OF MATHEMATICS \vspace{0.75mm}\newline\extraLight\scshape University of Belgrade}\newline

{\vspace{-3mm}\dateFont\extraLight Informatics, 240ECTS | Oct 2018 -- present}\newline

\vspace{-3mm}
{\contentFont I'm currently enrolled to 3rd year of Bachelor studies in Informatics, passing all examinations on schedule. You can find all my coursework on GitHub. % you can't, TODO upload and add link
During the course of my studies, I've developed a fully configurable ballistic projectile \cvLink{https://github.com/luka-j/ballistic-projectile/}{flight path simulator} in Python. I was also a part of the team which developed \emph{projekat iz RG goes here}.}
\end{minipage}\newline%
%
\begin{minipage}[t]{\linewidth}\vspace{\subsectionSpace}
{\subtitleFont\bold MATHEMATICAL GRAMMAR SCHOOL}\newline
{\vfill\vspace{\dateTopMargin}\dateFont\extraLight Sep 2014 -- Jun 2018\alignRight}\linebreak\newline

\vspace{\contentTopMargin}
{\contentFont I graduated from a school with High National Distinction status, specialized for mathematics and related disciplines, with 5/5 grade. My \cvLink{https://www.mg.edu.rs/uploads/files/images/stories/dokumenta/maturski/luka-jovicic.pdf}{graduation thesis} on data scraping and analysis of national primary school graduation results was selected as \cvLink{https://www.mg.edu.rs/sr/nastava/najbolji-maturski-radovi}{one of the best informatics theses} for school year 2017/18.}
\end{minipage}\newline%
%
% Extracurricular activities
%
\begin{minipage}[t]{\linewidth}\vspace{\sectionTopmargin}
{\titleFont\light EXTRACURRICULARS}\newline

\vspace{\titleBottomMargin}
{\subtitleFont\bold CS WEEK AT MGB \dateFont\extraLight(2018)}\newline
\newline

\vspace{-7mm}
{\contentFont A four-year regular participant, I held a lecture on exploratory data analysis at 2018 edition of \cvLink{http://csnedelja.mg.edu.rs/}{Computer Science Week at Mathematical Grammar School}. I've built a demo web app written in R \cvLink{https://github.com/luka-j/csw5-eda}{available on my GitHub}.}
\end{minipage}
\newline%
%
\begin{minipage}[t]{\linewidth}\vspace{\subsectionSpace}
{\subtitleFont\bold ENGLISH LANGUAGE CONFERENCES}\newline
\newline

\vspace{-7mm}
{\contentFont I've co-authored and presented papers (with high school teacher and peers) on Canadian film and teaching English as a second language at two international language conferences in \cvLink{https://www.npao.ni.ac.rs/filozofski-fakultet/download/391_964a4088cf25e02b20de75ae7a784ddf}{Belgrade (2017)} and \cvLink{http://fl.uni-mb.si/wp-content/uploads/2011/09/Book-of-Abstracts-ILC-SLO-2011.pdf}{Celje (2018)} respectively.}
\end{minipage}
% End left column
\end{minipage}%
%
%
% RIGHT COLUMN
%
%
\hfill
\begin{minipage}[t]{\rightColWidth}\vspace{\columnTopMargin}
%
% Tech
%
\begin{minipage}[t]{\linewidth}\vspace{\sectionTopmargin}
{\alignRight\titleFont\light TECH STACK}\linebreak\newline

\vspace{\contentTopMargin}
{\contentFont\bold Main toolbox: \medium Java, Spring Boot, Docker, Jenkins, PostgreSQL, Kotlin}

\vspace{2mm}
{\contentFont\bold Familiar with: \medium C, Bash, Python, R, Play! Framework, Android, WildFly, JavaScript, HTTP, \LaTeX}

\vspace{2mm}
{\contentFont\bold Experimenting with: \medium Scala, Redis \& Sentinel, Docker Swarm, distributed computing, Prolog, C++}

\vspace{2mm}
{\contentFont\bold Curious about: \medium Rust, Go, big data, GCP, neural networks}
\end{minipage}\newline%
%
% Informal education
%
\begin{minipage}[t]{\linewidth}\vspace{\sectionTopmargin}
{\alignRight\titleFont\light INFORMAL\linebreak\alignRight EDUCATION}\linebreak\newline
{\alignRight\rSubtitleFont\bold Petnica Summer Insitute}\newline
{\alignRight\rSubSubtitleFont\light Machine learning, 2019}\linebreak\newline

\vspace{\rContentTopMargin}
{\contentFont I've participated in an intensive 10-day bootcamp on machine learning, organized by Microsoft Development Center Serbia and Petnica SC. After partaking in lectures and workshops, I've worked on a project which used a conditional generative adversarial network to generate metal album cover arts. Our testimonial is featured \cvLink{http://psiml.petnica.rs/MetalGan.php}{on the PSI:ML website}.}

\end{minipage}\newline%
%
\begin{minipage}[t]{\linewidth}\vspace{\subsectionSpace}
{\alignRight\rSubtitleFont\bold Petnica Science Center}\newline
{\alignRight\rSubSubtitleFont\light Psychology seminars, 2017–18}\linebreak\newline

\vspace{\rContentTopMargin}
{\contentFont I've taken part in various lectures and workshops about research psychology. After writing practice papers and projects, I spent one year working on my own research on the role of emoticons and emojis in nonverbal irony, presented at the annual “Step Into Science” conference and published in Petnica Papers.}
\end{minipage}\newline%
%
\begin{minipage}[t]{\linewidth}\vspace{\subsectionSpace}
{\alignRight\rSubtitleFont\bold JCon}\newline
{\alignRight\rSubSubtitleFont\light Online conference, 2020}\linebreak\newline

\vspace{\rContentTopMargin}
{\contentFont I've attended one of the largest Java conferences, learning about the future of Java, trends in microservice development in 2020, best testing practices and pathways to cloud native apps. \cvLink{https://gist.github.com/luka-j/2c1230dea9f24c3a9b9cb30d661bb02e}{My notes are available on Gist.}}
\end{minipage}\newline%
%
% Scholarships
%   
\begin{minipage}[t]{\linewidth}\vspace{\sectionTopmargin}
{\alignRight\titleFont\light SCHOLARSHIPS}\linebreak\newline

\vspace{\rContentTopMargin}
{\contentFont\textbf{National scholarship for students}\linebreak (2019/2020, 2020/2021), awarded to higher education students which maintain exceptional GPA.}\alignRight\linebreak
{\contentFont\textbf{???} (2020/2021), ???.}
\end{minipage}\newline
% End right column
\end{minipage}
%
% End page 1
%
%
% PAGE 2
%
% Left column
\begin{minipage}[t]{\leftColWidth}\vspace{\pageTwoTopMargin}
%
% Selected projects
%
\begin{minipage}[t]{\linewidth}
{\titleFont\light SELECTED PROJECTS}\newline

\vspace{\titleBottomMargin}
{\subtitleFont\bold STORIES \extraLight\scshape| Library \& Android app \href{https://github.com/luka-j/stories}{\faGithub} \href{https://play.google.com/store/apps/details?id=rs.lukaj.android.stories}{\faAndroid}}\newline%There's no \faGooglePlay :/

\vspace{\ribbonTopMargin}
\ribbonJava \ribbonAndroid \ribbonPlay \ribbonPostgres\newline
\vspace{\ribbonBottomMargin}

{\contentFont App for creating, sharing and playing choose-your-adventure stories. I’ve designed a simple procedural language for storytelling and implemented a platform-agnostic interpreter (open source). Android app (proprietary) features a code editor, a visual story editor and an online marketplace. It won Special prize for innovativity at Telekom Srbija Regional App Challenge 2018.}
\end{minipage}\newline%
%
\begin{minipage}[t]{\linewidth}\vspace{\subsectionSpace}
{\subtitleFont\bold UPIS SUITE \extraLight\scshape| Desktop \& Web apps \href{https://github.com/luka-j/UpisScraper}{\faGithub} \href{https://github.com/luka-j/UpisStats}{\faGithub} \href{https://github.com/luka-j/UpisDesktop}{\faGithub}}\newline

\vspace{\ribbonTopMargin}
\ribbonJava \ribbonPlay \ribbonSwing \ribbonPostgres \ribbonJavascript\newline
\vspace{\ribbonBottomMargin}

{\contentFont Three apps designed to work together in order to collect and interpret data about Serbian primary schools. I've built a scraper which collects students’ grades and graduation data, enrolment simulator with support for custom rules, SQL-like data plotting language, web frontend for plots, performant plotter in Swing (able to handle and modify $~10^5$ points) and a visual query builder for desktop. Data and code is freely available on my GitHub.}
\end{minipage}\newline%
%
\begin{minipage}[t]{\linewidth}\vspace{\subsectionSpace}
{\subtitleFont\bold CHAT SURVEYS \extraLight\scshape| Web apps \href{https://github.com/Runtime-T-error}{\faGithub}}\newline

\vspace{\ribbonTopMargin}
\ribbonR \ribbonTidyverse \ribbonPostgres \ribbonJava \ribbonSpringBoot\newline
\vspace{\ribbonBottomMargin}

{\contentFont A set of apps which transform surveys to conversational format, using Facebook messages as proof of concept. My role was focused on interpreting existing engagement data and building dashbord for future analytics in R's shiny framework, supporting simple plots as well as inference methods such as linear regression and PCA. The project won \nth{3} prize at RAF Hakaton 2019.}
\end{minipage}\newline%
%
\begin{minipage}[t]{\linewidth}\vspace{\subsectionSpace}
{\subtitleFont\bold VIDEO COMPRESSOR \extraLight\scshape| Web app \href{https://github.com/luka-j/MediaCompressor}{\faGithub}}\newline

\vspace{\ribbonTopMargin}
\ribbonKotlin \ribbonSpringBoot \ribbonPostgres \ribbonFfmpeg\newline
\vspace{\ribbonBottomMargin}

{\contentFont A web app which transcodes videos using x265 codec and applies size optimizations for video and audio data. It can use one or more worker nodes to speed up the process, as well as act as a worker for configured master nodes.}
\end{minipage}\newline%
%
\begin{minipage}[t]{\linewidth}\vspace{\subsectionSpace}
{\subtitleFont\bold CV \extraLight\scshape| Interactive document \href{https://github.com/luka-j/CV}{\faGithub}}\newline

\vspace{\ribbonTopMargin}
\ribbonLatex \ribbonFontAwesome \newline
\vspace{\ribbonBottomMargin}

{\contentFont I've designed, made-up and built this very document on top of \LaTeX's \texttt{minimal} document class. It makes use of some PDF features which are not universally supported (e.g. in browsers)—best viewed in a full-fledged reader.}
\end{minipage}\newline
%
% EVEN MORE PROJECTS
%
\begin{minipage}{\linewidth}\vspace{5mm}
{\titleFont\light MORE PROJECTS}\vspace{1mm}\newline
{\subtitleFont \light CLICK THE \switchocg{HelpText}{\faChevronCircleRight}}\newline

\vspace{\titleBottomMargin}
{\subtitleFont\bold NOTEKEEPER \extraLight\scshape| Android app \href{https://play.google.com/store/apps/details?id=rs.luka.android.studygroup}{\faAndroid} \switchocg{NotekeeperText}{\faChevronCircleRight}}\newline

\vspace{\ribbonTopMargin}
\ribbonJava \ribbonAndroid \ribbonPlay \ribbonPostgres\newline
\end{minipage}\newline%
%
\begin{minipage}{\linewidth}
{\subtitleFont\bold\scshape MinimalHttpClient \extraLight\scshape| Java library \href{https://github.com/luka-j/MinimalHttpClient}{\faGithub} \switchocg{HttpClientText}{\faChevronCircleRight}}\newline

\vspace{\ribbonTopMargin}
\ribbonJava\newline
\end{minipage}\newline%
%
\begin{minipage}{\linewidth}
{\subtitleFont\bold LAB INVENTORY \extraLight\scshape| Web app \href{https://github.com/luka-j/laboratorija-inventar}{\faGithub} \switchocg{LabInventoryText}{\faChevronCircleRight}}\newline

\vspace{\ribbonTopMargin}
\ribbonKotlin \ribbonSpringBoot \ribbonPostgres \ribbonThymeleaf \newline
\end{minipage}\newline%
%
\begin{minipage}{\linewidth}
{\subtitleFont\bold\scshape BG BUS \extraLight\scshape| Android app \href{https://github.com/luka-j/BgBus}{\faGithub} \switchocg{BgBusText}{\faChevronCircleRight}}\newline

\vspace{\ribbonTopMargin}
\ribbonJava \ribbonAndroid \newline
\end{minipage}\newline%
% End left column
\end{minipage}
%
%
% RIGHT COLUMN
%
%
\hfill
\begin{minipage}[t]{\rightColWidth}\vspace{\pageTwoTopMargin}
%
% Competitons
%
\begin{minipage}[t]{\linewidth}
{\alignRight\titleFont\light COMPETITIONS}\linebreak\newline

{\alignRight\rSubtitleFont\bold RAF Hackathon \light (team, 2020)}\newline
{\alignRight\largeContentFont\medium \nth{3} prize for \cvLink{https://github.com/Runtime-T-error}{Chat Surveys}}\newline

{\alignRight\rSubtitleFont\bold WATFOI API Design \alignRight Workshop \light (2020)}\newline
{\alignRight\largeContentFont\medium \nth{1} prize for API design and spec}\newline

{\alignRight\rSubtitleFont\bold Telekom Srbija Regional App \alignRight Challenge \light (2018)}\newline
{\alignRight\largeContentFont\medium Special prize for \cvLink{https://play.google.com/store/apps/details?id=rs.lukaj.android.stories}{Stories}}\newline

{\alignRight\rSubtitleFont\bold RAF Challenge \light (2017)}\newline
{\alignRight\largeContentFont\medium Finalist with \cvLink{https://github.com/luka-j/UpisDesktop}{UpisDesktop}}\newline

{\alignRight\rSubtitleFont\bold MatHackathon \light (team, 2017)}\newline
{\alignRight\largeContentFont\medium Finalist with an Arduino smart \alignRight home app}\newline

{\alignRight\rSubtitleFont\bold mt:s App Competition \light (2016)}\newline
{\alignRight\largeContentFont\medium \nth{2} prize for \cvLink{https://play.google.com/store/apps/details?id=rs.luka.android.studygroup}{Notekeeper}}\newline

{\alignRight\rSubtitleFont\bold mt:s App Competition \light (2015)}\newline
{\alignRight\largeContentFont\medium \nth{2} prize for \cvLink{https://github.com/luka-j/BgBus}{BgBus}}\newline

{\alignRight\rSubtitleFont\bold National programming \alignRight competitions \light (2014--2017)}\newline
{\alignRight\largeContentFont\medium Qualified to the national level}\newline

{\alignRight\rSubtitleFont\bold National programming \alignRight competitions \light (2012--2014)}\newline
{\alignRight\largeContentFont\medium Qualified to Serbian Informatics \alignRight Olympiad}\newline

{\alignRight\rSubtitleFont\bold National programming \alignRight competitions \light (2011/12)}\newline
{\alignRight\largeContentFont\medium \nth{3} prize on National competition}\newline

{\parbox{\linewidth}{\color{white}\fontsize{8}{9}\selectfont Hi there! I just need some white padding here, so why not use it :) You see, rlap, combined with PDF OCGs, is \emph{very} awkward to work with. It seems to have no respect for vspaces, paddings, even phantom text. White text works, so---yay, I guess! You seem to be either quite bored or quite interested. How did I know? Well, you're reading white text, duh. Let me help---hit me up on LinkedIn or send me an email so we can talk about tech, software engineering career, school prospects, music, creative writing, science, psychology, languages,... anything, really (I'd put links here, but they'd break the invisibility). I hope you enjoyed reading this as much as I did making it (minus the frustration of aligning everything and rephrasing sentences so they'd fit "just right"). Seems like we're running of space here: see ya!}}
\end{minipage}\newline%
%
% More project descriptions
%
\rlap{\begin{ocg}{OCG 0}{HelpText}{0}
  {\parbox{\linewidth}{\contentFont Great, it works! You can use the \faChevronCircleRight \ next to a project name to see its description. Click the same arrow again to close. Or you can use this one: \hideocg{HelpText}{\faChevronCircleLeft}}}
\end{ocg}\newline}
\rlap{\begin{ocg}{OCG 1}{NotekeeperText}{0}
  {\parbox{\linewidth}{\contentFont Productivity and collaboration app for groups of students attending the same school. Provides a streamlined way for collecting and sharing notes, questions and multimedia content for different courses. Includes course organizer and exam calendar. The app won \nth{2} prize at Sixth mt:s App competition (2016).\hideocg{NotekeeperText}{\faChevronCircleLeft}}}
\end{ocg}\newline}
\rlap{\begin{ocg}{OCG 2}{HttpClientText}{0}
  {\parbox{\linewidth}{\contentFont Implementation of HTTP/1.1 protocol from the ground up, using plain TCP sockets in Java. Provides an option to manually configure connection pool and reuses sockets wherever appropriate. Exposes both blocking (with timeout) and non-blocking (callback-based) API. Provides common abstractions for sending and receiving textual, raw and file content. Transparently handles redirects, content and transfer encodings.\hideocg{HttpClientText}{\faChevronCircleLeft}}}
\end{ocg}\newline}
\rlap{\begin{ocg}{OCG 3}{LabInventoryText}{0}
  {\parbox{\linewidth}{\contentFont Web app for managing laboratory inventory, used in a local hospital. Tailored specifically for their needs, it tremendously speeds up inventory tracking and reporting for medical and non-medical supplies. UI was specifically built to be intuitive to non-technical users and print-friendly. \hideocg{LabInventoryText}{\faChevronCircleLeft}}}
\end{ocg}\newline}
\rlap{\begin{ocg}{OCG 4}{BgBusText}{0}
  {\parbox{\linewidth}{\contentFont Android app which simplifies navigation in Belgrade on foot and using public transport. It is built upon a few custom heuristic algorithms which take into account time of day, type of transport and how often it is deployed. After algorithms are run with different parameters, four best results are displayed to the user and drawn on the map. The app was awarded \nth{2} prize at Fifth mt:s App Competition (2015). \hideocg{BgBusText}{\faChevronCircleLeft}}}
\end{ocg}\newline}

\vspace{7mm}
{\contentFont Curious about even more projects? Check out \cvLink{https://github.com/luka-j?tab=repositories}{my GitHub!}}
% {\begin{ocg}{OCG 100}{DummyWhiteBox}{0}
%   \framebox(200,100){}
% \end{ocg}}
\end{minipage}
\end{document}
